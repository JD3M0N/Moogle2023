\section{Preaculo}
\subsection{Preaculo}

\begin{frame}
	\frametitle{Inicializando el Moogle}
	Explicacion de la realizacion del precalculo:
 Para el funcionamiento mas optimo de la aplicacion primero
 realizamos un precalculo de las operaciones que se puede ir
 tomando de antemano sin saber la consulta asi cuando el usuario
 decida realizar su consulta consumira menos tiempo. Para se ha
 creado el metodo Awake() que basta llamarlo antes de iniciar la
 aplicacion para que lea la carpeta content procesando los
 articulos y guardandolos en una estructura global y guardando
 tambien una estructura con todas las palabras del que
 aparecieron en el content y llamando al resto de metodos utiles
 para posteriormente procesar el modelo vectorial.


\end{frame}


\begin{frame}
	La estructura principal se basa en un Dictionary<string,
 Dictionary<string, double», donde el primer string seria el path al
 documento en el content como key y su valor seria un Dictionary
 con las palabras que tiene el documento como key acompanadas
 de la cantidad de veces que se repite esa palabra como value.
 El mismo llamado del la estructura principal aprovechando el
 poder de calculo del indexado de las palabras a los diccionarios
 de los documento se guarda un Diccionario global con todas las
 palabras que estan presetnes en el content con un value de la
 cantidad de documentos en los aparecio la palabra
 Posteeriormente se precalcula el Inverse Docuemnt Frequency
 (IDF), el Time Frequency (TF) y el Word Value Weight de la
 estructura principal para luego de realizada la consulta sea mas
 optima esta. Para el ultimo paso del precalculo se inicializa un
 Dataset de sinonimos para luego realizar una consulta mas
 optima buscando tambien los sinonimos de esta
\end{frame}


\begin{frame}
	
	$$
		IDF_{i}=\lg_{10}{(\frac{N}{n_{i}})}
	$$
	
	$$
		W_{ij}=\frac{freq_{ij}}{max_{l}freq_{lj}} IDF_{i}
	$$
\end{frame}

